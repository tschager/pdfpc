\documentclass{beamer}

\usepackage{calc}
\usepackage{hyperref}
\usepackage[utf8]{inputenc}
\usepackage{keystroke}
\usepackage{xspace}

\useoutertheme{shadow}
\usecolortheme{beaver}

\newcommand{\singleitem}[1]{\begin{itemize}\item #1\end{itemize}}
\newcommand{\pdfpc}{\texttt{pdfpc}\xspace}
\newcommand{\opt}[1]{\texttt{#1}\xspace}

\setbeamertemplate{navigation symbols}{}

\defbeamertemplate{footline}{my foot}{%
\vskip1pt%
\makebox[0pt][l]{\insertshortauthor}%
\hspace*{\fill}\insertshorttitle\hspace*{\fill}%
\llap{\insertframenumber\phantom{/}}
}
\setbeamertemplate{footline}[my foot]


\mode<presentation>

\title{\pdfpc demo}
\author[D. Vilar]{David Vilar \\ \texttt{davvil@gmail.com}}
\date{\today}
\institute{}

\begin{document}

\begin{frame}
  \titlepage
  \hypertarget{titlePage}{}
\end{frame}

\begin{frame}
  \frametitle{Starting up}
  Hi! Welcome to the demo of \pdfpc, a pdf presentation tool.
  \begin{block}{Setting up}
    I assume you have opened this presentation with \pdfpc itself. You may have
    one of these cases:
    \begin{itemize}
      \item The main presentation and the presenter view are on the correct
        screens
          \singleitem{you are ready to go!}
      \item The main presentation and the presenter view have the monitors
        swapped
          \singleitem{restart \pdfpc with the \opt{-s} option}
      \item You are viewing this on a single monitor
          \singleitem{you are currently on the presenter view. You can get a
          good feeling of what \pdfpc is good for, but you will have to take
          my word for some of the features}
    \end{itemize}
  \end{block}
\end{frame}

\begin{frame}
  \frametitle{The presenter view}
  The presenter view is the one on your monitor, visible only to the presenter
  \begin{itemize}
    \item It shows the current and next slides
    \item It shows how long the presentation is running
      \singleitem{With the \opt{-d <min>} option it shows a countdown timer}
    \item It shows the current slide number and the total number of slides
    \item It shows overlay information (see next slide)
    \item It shows some additional status information (we will see it later)
  \end{itemize}
  Of course, the presentation view only shows the current slide
\end{frame}

\begin{frame}
  \frametitle{Basic movement}
  \begin{itemize}
    \item You probably have guessed by now, but for moving forward you can use
      one of \RArrow, \PgDown, \Return (Enter), \Spacebar (space bar) and the left mouse
      button

      \alert{Important:} Do not use \DArrow (yet!) for moving forward
    \item Likewise you can use \LArrow, \PgUp or the right mouse button for
      going back

      \alert{Important:} Do not use \UArrow (yet!) for moving backwards
    \item Pressing shift in addition to one of the previous keys
      causes the movement to jump in 10-slide steps.
  \end{itemize}
\end{frame}

\begin{frame}
  \frametitle{Overlays}
  \begin{itemize}
    \item Some people like overlays in their presentations
    \pause
    \item i.e.\ slides that build up step-wise
    \pause
    \item \pdfpc supports such slides:
    \pause
    \item Note the two miniatures below the current slide
    \pause
    \item Note also that the slide counter shows the correct slide number
    \pause
    \item The 10-jump slides mentioned in the previous slide also support
      overlays
    \pause
    \item If you are fed up with this overlay, you can jump to the next slide
      with \DArrow
    \pause
    \item Really you should skip to the next slide now
    \pause
    \item Nothing interesting is coming here
    \pause
    \item I just wanted to make a point that too long overlays may be boring
    \pause
    \item Come on, nothing to see here!
    \pause
    \item I am not kidding!
    \pause
    \item Ok, you won
  \end{itemize}
\end{frame}

\begin{frame}
  \frametitle{More on overlays}
  \begin{block}{Movement}
    \begin{itemize}
      \item Normal movement while \emph{inside} an overlay jumps through each step
        in the overlay
      \item Movement backwards \emph{outside} of an overlay jumps to the beginning
        of the overlay
        \singleitem{e.g.\ press \PgUp now}
      \item \UArrow and \DArrow jump to the previous and next slide respectively,
        ignoring overlay steps
    \end{itemize}
  \end{block}
  \begin{block}{Definition of overlays}
    \begin{itemize}
      \item \pdfpc tries to find the overlay information automatically by
        looking at the page numbers
      \item If automatic detection does not work, you can define overlays with
        \keystroke{O} on every slide composing an overlay
    \end{itemize}
  \end{block}
\end{frame}

\begin{frame}
  \frametitle{Jumping to specific slides}
  \begin{itemize}
    \item Pressing \keystroke{G} you can enter a slide number to jump to
      \singleitem{To come back to this slide you can use the \BSpace (backspace) key or remember that we are on slide \insertframenumber.}
    \item Pressing \Tab (Tab) you get an overview of the whole presentation,
      which you can use to jump around (note that only the first overlay slide
      is shown in the overview!)
    \item Hyperlinks also work, press \hyperlink{titlePage}{here to jump to the
      first slide}
    \item \Home and \End also work as expected
  \end{itemize}
\end{frame}

\begin{frame}
  \frametitle{Notes}
  \begin{itemize}
    \item Pressing \keystroke{N} you can annotate slides (text only)
      \singleitem{Try it now!}
    \item Pressing \Esc exits the note editing mode
    \item The notes are stored automatically
    \item The notes apply to all the slides in an overlay
  \end{itemize}
\end{frame}

\begin{frame}
  \frametitle{Controlling the presentation view}
    \begin{itemize}
      \item Pressing \keystroke{F} you can freeze the presentation view, i.e.\
        slide movement does not reflect on the presentation view
        \singleitem{This is useful if you want to search for some slide without
        confusing the audience with quick slide flipping}
      \item Pressing \keystroke{B} you can fade to black the presentation view
        \singleitem{This is useful e.g.\ if you are using slides and a
        blackboard at the same time}
      \item The presenter view shows some icons reflecting the current status
    \end{itemize}
\end{frame}

\begin{frame}
  \frametitle{Controlling the clock}
  \begin{itemize}
    \item With \keystroke{S} you can start the clock at the beginning of the
      presentation
      \singleitem{Note that the clock also starts automatically when moving slides}
    \item With \keystroke{P} you can pause the clock
      \singleitem{Useful for rehearsal talks}
    \item With \keystroke{R} you can reset the presentation
      \singleitem{i.e.\ set the clock to 0 and jump to the first slide}
  \end{itemize}
\end{frame}

\begin{frame}
  \frametitle{Finishing}
  \begin{itemize}
    \item Some people like having support slides after their ``last'' slide
      \singleitem{Advanced topics, bilbiography, expected questions, etc.}
    \item Pressing \keystroke{E} you can define this to be the end slide in the
      presentation
      \singleitem{The slide count in the presenter view gets updated, to give you
      a better overview of how many slides are left}
      \singleitem{It also updates the slide you jump to when pressing \End}
    \item For exiting, press \keystroke{Q} or \Esc
  \end{itemize}
  \vfill
  \begin{center}
    THE END
  \end{center}
  \vfill
  {\footnotesize There are a couple of slides more with some additional information,
  but we do not want to show them to the audience if they do not ask for them}
\end{frame}

\begin{frame}
  \frametitle{\opt{.pdfpc} files}
  \begin{itemize}
    \item The additional information needed for the presentation (duration,
      notes, end slide, etc.) is stored in an additional file with extension
      \opt{.pdfpc}
    \item Most of the time this file is automatically handled
    \item If you ever need to do changes by hand (e.g.\ if you modify the pdf
      after defining the meta-information) it is a text-based format easy to
      edit
  \end{itemize}
\end{frame}

\begin{frame}[fragile]
  \frametitle{Acknowledgements}
  \begin{itemize}
    \item \pdfpc is a fork of pdf-presenter-console
      ({\footnotesize \verb+http://westhoffswelt.de/projects/pdf_presenter_console.html+})
    \item Many thanks to Jakob Westhoff, the original author!
  \end{itemize}
\end{frame}

\end{document}
